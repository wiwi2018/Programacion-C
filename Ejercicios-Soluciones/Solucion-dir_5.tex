\documentclass[twoside,10.5pt]{article}%
\usepackage{minted}   % importamos el paquete minted                       
\usepackage{mathrsfs}% 
\usepackage[sc]{mathpazo}                                          
\usepackage{pifont}%                                             
\usepackage{amsmath}%                                            
\usepackage{amsthm}%                                             
\usepackage{txfonts}%                                            
\usepackage{geometry}%                                           
\usepackage{latexsym}%                                           
\usepackage{amssymb}%                                            
\usepackage{graphicx}%                                           
\usepackage{geometry}%                                           
\usepackage{xcolor} %                                            
\geometry{paperheight=28.5cm,paperwidth=21cm,top=2.5cm,%         
bottom=2.6cm,left=2.5cm,right=2.5cm,headheight=0.8cm,%             
headsep=0.9cm,textheight=20cm,footskip=1cm}%                   
\setlength{\parindent}{0pt} \setlength{\parskip}{5pt}%           
\renewcommand{\baselinestretch}{1.0}%                            *                                                        *
\pagestyle{empty}
\begin{document}
\begin{center}
{\LARGE{Soluciones de la dirigida 5}}\\[20pt]
\end{center}

\vspace{0.3cm}




\begin{minted}{c}
#include<stdio.h>

int inicio(void); // Prototipo de la función universo
int muestra(int n); // Prototipo de la función muestra
int combinatoria(int n, int k); // Prototipo de la función combinatoria

int main()
{
	int n,k; 
	n = inicio();	
	k = muestra(n);
	printf("El n\'umero de combinaciones de %d
	 elemento(s) de %d elemento(s) dado(s) es %d\n",k,n,combinatoria(n,k));	
}

int inicio(void) // cabecera de la definición universo
{
	int ok = 0; 
	int n; // variable donde almacenar\'e lo que voy a retornar
	printf("Ingrese un entero positivo: "); 
	scanf("%d",&n);
	return n;
}

int muestra(int n) // cabecera de la definición de la funci\'on muestra
{
	int ok = 0; 
	int k; // variable donde almacenar\'e lo que voy a retornar
	printf("Ingrese un entero positivo menor o igual a %d: ",n); 
	scanf("%d",&k);
	return k;
}
		
int combinatoria(int n, int k) // cabecera de la definición de la función combinatoria
{
	if (k == 1)	
		return n;
	else
		if (n == k)
			return 1;
		else
			return combinatoria(n-1,k-1) + combinatoria(n-1,k);
}
\end{minted}


\vspace{0.3cm}


\begin{minted}{c}
#include<stdio.h>

int ingresar(void); // Prototipo de la función ingresar
int fibo(int n); // Prototipo de la función fibo

void main()
{
	int n; 
	n = ingresar();
	printf(" F(%d) = %d.\n\n", n, fibo(n));
}

int ingresar(void) // cabecera de la definición de la funci\'on exponente
{
	int n; // variable donde almacenar\'e lo que voy a retornar
	printf("\n Ingrese el exponente (n\'umero entero): "); 
	scanf("%d",&n);
	return n;
}
	
int fibo(int n) // cabecera de la definición de la función fibo
{ 
	if ( n <= 2 ) 
		return 1; // fibo(1) = fibo (2) = 1
	else
		return fibo(n-1) + fibo(n-2); // fibo(n) = fibo(n-1) + fibo(n-2)
}
\end{minted}


\vspace{0.5cm}


\begin{minted}{c}
#include<stdio.h>

float base(void); // Prototipo de la función base
int exponente(void); // Prototipo de la función exponente
float potencia(float b, int n); // Prototipo de la función potencia

int main()
{
	float b; 
	int n; 
	b = base();	
	n = exponente();
	printf("%f elevado a la %d es igual a %f\n",b,n,potencia(b,n));	
}

float base(void) // cabecera de la definición base
{
	int ok = 0; 
	float b; // variable donde almacenar\'e lo que voy a retornar
	printf("Ingrese la base (n\'umero positivo): "); 
	scanf("%f",&b);
	return b;
}

int exponente(void) // cabecera de la definición de la funci\'on exponente
{
	int n; // variable donde almacenar\'e lo que voy a retornar
	printf("Ingrese el exponente (n\'umero entero): "); 
	scanf("%d",&n);
	return n;
}
	
float potencia(float b, int n) // cabecera de la definición de la función potencia
{
	float retorno = 1;
	if ( n > 0 )
		retorno = b*potencia(b,n-1);
	if ( n < 0 )
		retorno = potencia(b,n+1)/b;
	return retorno;
}
\end{minted}


\vspace{0.5cm}

\begin{minted}{c}
/*
 * Un programa en C para encontrar el MCD de dos 
  numeros usando recursion
 */
#include <stdio.h>
 
int gcd(int, int);
 
int main()
{
    int a, b, resultado;
 
    printf(" Ingresa dos numeros para encontrar el MCD:");
    scanf("%d%d", &a, &b);
    resultado = mcd(a, b);
    printf("El MCD  DE  %d y  %d es %d.\n", a, b, resultado);
}
 
int mcd(int a, int b)
{
    while (a != b)
    {
        if (a > b)
        {
            return mcd(a - b, b);
        }
        else
        {
            return mcd(a, b - a);
        }
    }
    return a;
}
\end{minted}

\vspace{0.5cm}


\begin{minted}{c}
/*
 * Un programa para revertir un numero
 */
#include <stdio.h>
#include <math.h>
 
int rev(int, int);
 
int main()
{
    int num, resultado;
    int longitud = 0, temp;
 
    printf("Ingresar un numero entero para revertir: ");
    scanf("%d", &num);
    temp = num;
    while (temp != 0)
    {
        longitud++;
        temp = temp / 10;
    }
    resultado = rev(num, longitud);
    printf("El reverso de  %d es %d.\n", num, resultado);
    return 0;
}
 
int rev(int num, int len)
{
    if (len == 1)
    {
        return num;
    }
    else
    {
        return (((num % 10) * pow(10, len - 1)) + rev(num / 10, --len));
    }
}
\end{minted}
\end{document}