\documentclass[twoside,10.5pt]{article}%
\usepackage{minted}   % import the package here                         
\usepackage{mathrsfs}% 
\usepackage[sc]{mathpazo}                                          
\usepackage{pifont}%                                             
\usepackage{amsmath}%                                            
\usepackage{amsthm}%                                             
\usepackage{txfonts}%                                            
\usepackage{geometry}%                                           
\usepackage{latexsym}%                                           
\usepackage{amssymb}%                                            
\usepackage{graphicx}%                                           
\usepackage{geometry}%                                           
\usepackage{xcolor} %                                            
\geometry{paperheight=28.5cm,paperwidth=21cm,top=2.5cm,%         
bottom=2.6cm,left=2.5cm,right=2.5cm,headheight=0.8cm,%             
headsep=0.9cm,textheight=20cm,footskip=1cm}%                   
\setlength{\parindent}{0pt} \setlength{\parskip}{5pt}%           
\renewcommand{\baselinestretch}{1.0}%                            *                                                        *
\pagestyle{empty}
\begin{document}
\begin{center}
{\LARGE{Algunas soluciones de la  dirigida 7}}\\[20pt]
\end{center}

\vspace{0.3cm}

\begin{enumerate}
\item Podemos resolver el problema, por ejemplo de dos maneras

\begin{minted}{c}
#include <stdio.h>
 
int main()
{
   int x, y, *a, *b, temp;
 
   printf("Ingresa los valores de x y y \n");
   scanf("%d%d", &x, &y);
 
   printf("antes del intercambio \nx = %d\ny = %d\n", x, y);
 
   a = &x;
   b = &y;
 
   temp = *b;
   *b   = *a;
   *a   = temp;
 
   printf("Despues del intercambio\nx = %d\ny = %d\n", x, y);
 
   return 0;
}
\end{minted}

Otra forma es la siguiente 

\begin{minted}{c}
/* Program para intercambiar dos numeros dados 
*/

#include<stdio.h>
 
void  intercambio(int *num1, int *num2) {
   int temp;
   temp = *num1;
   *num1 = *num2;
   *num2 = temp;
}
 
int main() {
   int num1, num2;
 
   printf("Ingresa el primer numero:\n ");
   scanf("%d", &num1);
   printf("Ingresa el segundo numero:\n ");
   scanf("%d", &num2);
 
   intercambio(&num1, &num2);
 
   printf("Primer numero : %d\n", num1);
   printf("Segundo numero : %d\n", num2);
 
   return  0;
}
\end{minted}


\item Escribe un programa que encuentre el menor y mayor elemento de un  arreglo num\'erico.

\begin{minted}{c}
/* Encuentra el mayor menos valor de un array numerico */


#include <stdio.h>

#define N 10


void max_min(int a[], int n, int *max, int *min);

int main(void)
{ 
  
  int b[N], i,  mayor, menor;
   printf("Ingresa los  %d numeros:", N);
     for (i = 0; i < N; i ++)
     scanf("%d", &b[i]);
     max_min(b, N, &mayor, &menor);
    printf("Mayor: %d\n", mayor);
    printf("Menor: %d\n",  menor);
     return 0;
}

void max_min(int a[],  int n, int *max,int *min)
  {
     int i;
     *max = *min = a[0];
     for (i = 1; i < n; i ++){
       if (a[i] > *max) 
	 *max = a[i];
       else if (a[i] < *min)
	 *min = a[i];
     }
  }
\end{minted}

\item Escribamos un programa que realiza las operaciones matem\'aticas usuales, con dos n\'umeros ingresados por el usuario. 



\begin{minted}{c}
/*
 * Operaciones con punteros
 */

#include <stdio.h>
#include <stdlib.h>
#include <math.h>
double Sum(double x, double y){return x + y;}
double Sub(double x, double y){return x - y;}
double Mul(double x, double y){return x * y;}
double Div(double x, double y){return x / y;}

// Arreglo de  5 punteros a  funciones que toma doubles como parametro 
// y retorna double :
double (*funcTabla[5])(double, double)
    = { Sum, Sub, Mul, Div, pow }; // Lista a inicializar.

 //Un array de punteros a cadenas para la salida: 
char *msgTabla[5] = { "Suma", "Diferencia", "Producto", "Cociente", "Potencia" };

int main( )
{
   int i;
   double x = 0, y = 0;
// Una variable indice
   
  printf( "Ingresa dos operandos para alguna operacion: :\n" );
 if ( scanf( "%lf %lf", &x, &y ) != 2 )
     printf( "Entrada invalida.\n" );
 for ( i = 0; i < 5; ++i )
  printf( "%10s: %6.2f\n", msgTabla[i], funcTabla[i](x, y) );
      return 0;
}
\end{minted}

\item Escribamos un programa, en cada cadena ingresada, sea o no un planeta

\begin{minted}{c}
/*
 * Verificando el nombre de los planetas
 */

#include <stdio.h>
#include <string.h>

#define NUM_PLANETAS 9
int main(int argc, char *argv[])
{
  
  char *planetas[] = {"Mercurio", "Venus", "Tierra",
                   "Marte", "Jupiter", "Saturno", "Urano", 
		   "Neptune","P1uton"};
int i, j;
for (i = 1; i<argc; i++){
  
  for (j = 0; j< NUM_PLANETAS; j++)
      if (strcmp(argv[i], planetas[j]) == 0) {
	printf("%s es una planeta %d\n", argv[i], j + 1);
	  break;
	}
if (j == NUM_PLANETAS)
      printf("%s no es un planeta\n", argv[i]);
    }
return 0;
}
\end{minted}

\item Desarrollemos un programa que calcule promedios. Usamos como ejemplo funciones lineales y cuadr\'aticas 

\begin{minted}{c}
/* Programa que puede calcular el promedio 
de cualquier funcion recibida como 
parametros
*/

#include <stdio.h>

double promedio(double (*func)(int), int inicio, int fin)
{
  double acum = 0.0;
  int x;
  
  for(x = inicio; x < fin; x++)
    acum += (*func)(x);
  return acum/(fin -inicio);
}

/*Funcion Lineal */

double lineal(int x)
{
  return (double)x;
}

/* Funcion cuadratica */

double cuadratica(int x)
{
  return (double)x*x;
}

int main()
{
  printf("Promedio de f(x) = x (0 <= x< 10) : %.2f\n", promedio(lineal, 0,10));
  printf("Promedio de f(x) = x^2 (3 <= x< 17) : %.2f\n", promedio(cuadratica, 3,7));
  
  return 0;
}
\end{minted}

\item Escribamos un programa en C, que inserte una subcadena en una cadena.


\begin{minted}{c}
/* Programa para ingresar una subcadena
 * dentro de una cadena
 */

#include <stdio.h>
#include <string.h>
#include <stdlib.h>
 
void insert_subcadena(char*, char*, int);
char* subcadena(char*, int, int);
 
int main()
{
   char text[100], subcadena[100];
   int posicion;
 
   printf("Ingrese algun texto\n");
   fgets(text,sizeof(text),stdin);
 
   printf("Ingrese la subcadena a ingresar\n");
   fgets(subcadena, sizeof(subcadena), stdin);
 
   printf("Ingrese la posicion donde quiera poner la subcadena\n");
   scanf("%d", &posicion);
 
   insert_subcadena(text, subcadena, posicion);
 
   printf("%s\n",text);
 
   return 0;
}
 
void insert_subcadena(char *a, char *b, int posicion)
{
   char *f, *e;
   int longitud;
 
   longitud = strlen(a);
 
   f = subcadena(a, 1, posicion - 1 );      
   e = subcadena(a, posicion, longitud-posicion+1);
 
   strcpy(a, "");
   strcat(a, f);
   free(f);
   strcat(a, b);
   strcat(a, e);
   free(e);
}
 
char *subcadena(char *cadena, int posicion, int longitud) 
{
   char *pointer;
   int c;
 
   pointer = malloc(longitud+1);
 
   if( pointer == NULL )
       exit(EXIT_FAILURE);
 
   for( c = 0 ; c < longitud ; c++ ) 
      *(pointer+c) = *((cadena+posicion-1)+c);       
 
   *(pointer+c) = '\0';
 
   return pointer;
}
\end{minted}
\item Escribamos un programa que devuelve un n\'umero en orden inverso.

\begin{minted}{c}
/* Programa que retorna el inverso de un numero
*/
#include <stdio.h>

#define N 10

int main (void)

{
     int a[N], *p;
     
    printf("Ingresa los %d numeros: ", N);
    for (p = a; p < a + N; p++)
     scanf("%d", p);
      printf("El numero en orden inverso:");

      for (p= a + N - 1; p>= a; p--)
      printf("%d ", *p);
      printf("\n");
      
      return 0;
}
\end{minted}



\item Escribamos un programa en C, que ordene una cadena en orden alfab\'etico. El programa s\'olo acepta cadenas en letras min\'usculas.

\begin{minted}{c}
/*
 * Programa que ordena una matriz alfabeticamente 
 */

#include <stdio.h>
#include <stdlib.h>
#include <string.h>
 
void ordenar_cadena(char*);
 
int main()
{
   char cadena[100];
 
   printf("Enter some text\n");
   fgets(cadena, sizeof(cadena), stdin);
 
   ordenar_cadena(cadena);
   printf("%s\n", cadena);
 
   return 0;
}
 
void ordenar_cadena(char *s)
{
   int c, d = 0, longitud;
   char *pointer, *resultado, ch;
 
   longitud = strlen(s);
 
   resultado = (char*)malloc(longitud+1);
 
   pointer = s;
 
   for ( ch = 'a' ; ch <= 'z' ; ch++ )
   {
      for ( c = 0 ; c < longitud ; c++ )
      {
         if ( *pointer == ch )
         {
            *(resultado+d) = *pointer;
            d++;
         }
         pointer++;
      }
      pointer = s;
   }
   *(resultado+d) = '\0';
 
   strcpy(s, resultado);
   free(resultado);
}
\end{minted}


\item Escribamos la Criba de Erat\'ostenes, usando punteros

\begin{minted}{c}
/*
** Criba de Eratostenes –– calculamos numeros primos usando un arreglo.
*/

#include <stdlib.h>
#define SIZE 1000
#define TRUE 1
#define FALSE 0

int   main()
{
  char   criba[SIZE]; // La criba
  char *sp;       // El puntero accede a la criba
  int numero;     // numero que estamos calculando
 
/* Ponemos toda la criba a TRUE */
   for( sp = criba; sp < &criba[ SIZE ]; )
     *sp++ = TRUE;

/*
Procesamos cada numero desde 3  hasta el numero 
de la criba 
*/
   
for( numero = 3; ; numero += 2 ){

  /*
   *Ponemos el puntero al elemento propio de la criba
   *paramos si nos pasamos  
   */
  sp = &criba[0] + (numero-3 )/2;
  if( sp >= &criba[ SIZE ] )
      break;
/*
 * Ahora avanzamos el puntero por multiplos del numero
 * y colocamos cada subsucesion FALSE.
*/
while( sp += numero, sp < &criba[ SIZE ] )
   *sp = FALSE;
}
/*
* Vamos a traves de la criba y imprimimos los
*  numeros correspondiendo a las localizaciones que 
* permanecen TRUE.
*/
printf("2\n");
for( numero = 3, sp = &criba[ 0 ];
  sp < &criba[ SIZE ]; numero += 2, sp++ ){
    if( *sp )
      printf( "%d\n", numero );
    }
  return EXIT_SUCCESS;
}
\end{minted}


\item Escribe un programa en C, que muestra una cadena invertida, a partir de una cadena \mbox{ingresada} por el usuario. Yo uso la \texttt{gets()}\dots. La explicacion del uso de \texttt{fgets()} se har\'a en la clase.

\begin{minted}{c}
/*
 * Programa para revertir una cadena en C
 */


#include<stdio.h>
 
int cadena_longitud(char*);
void reverse(char*);
 
main() 
{
   char cadena[100];
 
   printf("Ingresa una cadena \n");
   //gets(cadena);
   fgets(cadena, sizeof(cadena), stdin);
    
 
   reverse(cadena);
 
   printf("La reversa de la cadena ingresada es  \"%s\".\n", cadena);
 
   return 0;
}
 
void reverse(char *cadena) 
{
   int longitud, c;
   char *inicio, *end, temp;
 
   longitud = cadena_longitud(cadena);
   inicio  = cadena;
   end    = cadena;
 
   for (c = 0; c < longitud - 1; c++)
      end++;
 
   for (c = 0; c < longitud/2; c++)
   {        
      temp   = *end;
      *end   = *inicio;
      *inicio = temp;
 
      inicio++;
      end--;
   }
}
 
int cadena_longitud(char *pointer)
{
   int c = 0;
 
   while( *(pointer + c) != '\0' )
      c++;
 
   return c;
}
\end{minted}
\end{enumerate}
\end{document}